\chapter{Background}

\section{Cryptographic Construction Types}

\subsection{Cryptographic Primitives "Does this go here or under PKC" Public Key}

\subsubsection{Problem solved by PKC}

Public Key Cryptography (PKC) solves the problem of how to enable two people who have never met to but need to communicate securely over insecure channels. 
Public-key uses a one-way function as its public key; this function is easy to compute but hard to invert and is available publicly for use as an encryption method for the key owner. A second key called the private key is known only to the
key owner; this key is a trapdoor function which can invert a ciphertext back to its plaintext form. 

Public-key is bidirectional. The forward direction is the one-way public key which easily computes the ciphertext, the backward direction is the trapdoor private key,
which easily inverts to the plaintext. 
\textbf{insert PKC Diagram}

\subsubsection{Asymmetric & Symmetric Cryptography} 

The concepts of symmetric and asymmetric cryptographic-key schemes are given below as independent forms. However, it is rarely the case that these schemes will be applied in this manner particularly in a Public-key asymmetric system.

\subsubsubsection{Symmetric-key Cryptography} 

In cannonical symmetric-key schemes only one key is generated for both the encryption and decryption algorithms. Symmetric-keys must therefore be exchanged over secure channels where both communicating parties have agreed in advance to the method of
encryption and have exchanged secret keys. 
\newline
Formally, we say the two parties are in possesion of a \textit{shared secret} and each party is mutually, as well as equally responsible for the maintenance of the secret which establishes and secures their communications. 
In certain contexts this method can be made to be as secure as an asymmetric key system its main drawback is the shared secret which has the ability to invalidate the integrity of communications if either party is compromised.
\newline
Additionally since the method requires a secure channel as well as a shared secret both parties must know they will have a need for encrypted communication in advance and negotiate the means of transmission and key exchange prior to communicating. This means the parties must know each other prior to communicating. 

\subsubsubsection{Asymmetric-key Cryptography}

In contrast symmetric-key schemes are determined by two sets of keys one for each party. Each set consists of a private key known only to the owner and a public key which can be widely distributed. It seems at first glance that asymmetric and symmetric cryptosystems are isolated methods of encryption. However, symmetric
encryption is well-suited for the task of assisting asymmetric key
schemes not only for efficiency purposes but also in the role of key
management. 

\section{Provable Security} 

\textbf{SHOULD POSSIBLY MERGE WITH THE HARDNESS ASSUMPTIONS SECTION} 

\subsection{Defining Security with Games} 

\subsubsection{Semantic Security: Mental Poker and Partial Information} 

Semantic Security was a game proposed by Goldwasser and Micali in their 1982 paper entitled: \textit{"Probabilistic Encryption /& How to Play Mental Poker Keeping Secret All Partial Information"} 
\newline
\textbf{INSERT CITE-KEY:} 
%@inproceedings{Goldwasser:1982:PEA:800070.802212, author = {Goldwasser, Shafi and Micali, Silvio}, 
%title = {Probabilistic Encryption /& How to Play Mental Poker Keeping Secret All Partial Information}, 
%booktitle= {Proceedings of the Fourteenth Annual ACM Symposium on Theory of Computing}, 
%series = {STOC '82}, 
%year ={1982}, 
%isbn = {0-89791-070-2},
%location = {San Francisco, California, USA}, 
%pages = {365--377},
%numpages = {13}, 
%url = {http://doi.acm.org/10.1145/800070.802212}, 
%doi= {10.1145/800070.802212}, 
%acmid = {802212}, 
%publisher = {ACM}, 
%address= {New York, NY, USA}, }} 
\newline
The Mental Poker game was based on the implementation of the \textbf{TO-DO add gloss} Diffie-Hellman implementation of RSA by Rivest, Shamir, Adelman, and Rabin.
\footnote{Specifically, [Rab1979] Rabin's 1979 Technical Memo \textit{"Digitalized Signatures and Public-key Functions as Intractable as Factorization"}}
\footnote{[Riv1978] Rivest, Shamir, and Adelmans paper from Publications of the ACM, February 1978 \textit{"A Method for Obtaining Digital Signatures and Public Key Cryptosystems."}}
\textbf{TO-DO: INSERT CITATIONS TO REFS: [Rab1979], [Riv1978], [Gol1982]}

\textbf{Formal Definition:}
\newline
[Gol1982] Proposed the following property for any implementation of a Diffie-Hellman Public-key Cryptosystem:  \textit{"An adversary, who knows the encryption of an algorithm and is given the cypher text, cannot obtain any information about the cleartext."} \footnote{[Gol1982] Pg.1, Paragraph 1, Lines 3-6} 

\newline

\textbf{Informally:}
\newline
A given cryptographic scheme is considered insecure if it is possible for an adversary to recover any information about the plaintext, using the ciphertext, but without knowing the private key. That is, if it is feasible for the adversary to find out some information about the plaintext of the message or recover useful information about the plaintext of the message by manipulating the ciphertext in a reasonable amount of time, 
%(before he dies of old age or the heat death of the universe occurs) then the cryptosystem that created those ciphertext messages (is broke as shit) is insecure. \textit{Kristi fix this wtf dude…}

\subsubsection{Weaknesses in the Assumptions of RSA} 
\newline

Goldwasser and Micali pointed out that the security assumptions given in [Riv1978] and [Rab1979] had some particularly significant weaknesses that could not be assumed lightly. Namely,

\begin{itemize} 
\item \textbf{Assumption 1: Security of the RSA system is based on the intractability of the number theoretic problems of factorization, index finding, and the decision problem of whether or not a number is a quadratic residue with respect to a composite modulus. \newline This assumption states that the impossible hardness of one of these problems is equivalent to RSA being computationally infeasible. 
\item \textbf{Assumption 2:} The second assumption is that there exists a trapdoor function $f(x)$ that is easily computed, while $x$ is not easily computed from $f(x)$ unless some additional information is known. \newline However, Not all hard problems are as hard as other hard problems. \newline In particular, RSA is based on the assumption that factoring large composites is a hard problem. \newline However, not all large composite numbers are hard to factor in fact some are quite easy. For example, if $c$ is a composite such that $c=qp$ for primes $p$ and $q$ then if $p$ and $q$ are close enough together we have some assurances they will be easily factorable. 
\newline
\item \textbf{TODO: INSERT FORMULA}
\newline
\end{itemize} 

\section{Equivalent and Stronger Definitions for Security} 

\textbf{INSERT INTRODUCTORY PARAGRAPH}

\begin{itemize}
\item \textbf{IND-CPA} IND-CPA says if an attacker can choose any plaintext and obtain the corresponding ciphertext, then if the system is secure this information does not help them find the private key. However, the issue with CPA is that it depends on the choices of an adversary who is unaware of the secret key. i.e. If the
attacker has two messages then they have no idea if either one contains the key since they don't know what the key is. 
\newline 
In application this classififcation is not sufficient (weakly sufficient) to ensure secure communication. In practice this becomes an issue when a user decides to encrypt their own key.
\newline 
By IND-CPA the user can encrypt their private-key but the scheme may return the private-key unencrypted. 
\newline 
By the definition of IND-CPA this action is still classified as secure, even though the scheme has blatantly revealed the users private key.
\newline

\textbf{TODO: INSERT MATHS DEFINITION AND FORMULA}

\item \textbf{IND-CCAI} 
\textbf{TODO: INSERT SECTION}
\item \textbf{IND-CCAII}
\textbf{TODO: INSERT SECTION}
\end{itemize}

\section{InfoSec /& it's Objectives} 

\subsection{Base Security Objectives}

There are four basic security objectives that must be considered when constructing any system concerned with securing data or information. These four basic definitions allow for the derivation of all other security objectives which may or may not be necessary to ensure the security of information for a given system.

\begin{itemize}
\item \textbf{Confidentiality:} The objective of confidentiality ensures that unauthorized users will not be purposefully (or accidentally) give access to resources protected by the system. 
\item \textbf{Integrity:} Ensures that the resources are preserved, used, and appropriately maintained throughout their life-cycle under the system. That is, any data is not alterable in an undetectable manner, retains the same
accuracy as its created date (or registered modification date), and is complete with respect to its creation and activity log. 
\item \textbf{Availability:}
\item \textbf{Non-repudiation:} 
\end{itemize}

\subsubsection{Derived Security Objectives}

Each implementation of a scheme requires a flexible set of security objectives which are dependent upon the context the system and it's users will employ \textbf{TODO:change wording}. The base definitions for the objectives allow the derivation of all other security objectives.

\newline
\textbf{TODO: INSERT TABLE (it is in figures)}
\newline
\section{Formal Security Reduction} 
\textbf{TODO: INSERT SECTION}
\textbf{Definition}
\textbf{TODO: INSERT SECTION}
\textbf{TODO: INSERT FORMULA}

\section{UNMODIFIED PLAINTEXT}
First Half Introduction Introducing the Game Modern cryptographic
schemes can be viewed as two classes asymmetric-key schemes and
symmetric-key schemes. Symmetric schemes are in general more efficient
than asymetric schemes but have the disadvantage that keys must be
agreed upon by parties who wish to communicate in advance. Specifically
symmetric-key schemes require that secret keys be distributed over
secure channels. These types of key distribution schemes are not
compatible with most modern methods of communication over networks,
which are by and large conducted over the internet, an insecure
channel. Public-key cryptography is an asymmetric-key class of systems
which allows parties which have never met, much less agreed upon secret
keys in advance, to communicate privately over insecure channels (we
continue this discussion in the background section of the text,
briefly). The primary concern is to make communication by public key
cryptographic schemes as efficient, secure, and easy to implement as
those of asymmetric key systems. The problem of mirroring the
characteristics of asymmetric key schemes is a difficult problem and
can be considered a benchmark for implementation of public-key schemes.
In this context use of symmetric encryption methods is often a second
stage problem. That is, first the cryptosystem is created with some
thought in mind to its implementation and efficiency but with more
thought in mind to its security, particularly in the face of incoming
or expected threats in the near future. Since the primary reason
cryptography exists is to keep communications and information private
the role of risk management is weighted necessarily heavier towards
ensuring the fulfillment of a given set of security objectives. An
encryption scheme however is useless unless it can be implemented and a
balance between efficiency and security must be carefully, and
continuously maintained throughout not only the development of the
scheme but during its implementation as well. The balancing act between
efficiency and security is illustrated in the first half of this paper
which gives an account of the events and results following an
unexpected announcement by British researchers in the
Communications-Electronics Security Group (CESG) (the information
assurance division of the Government Communications Headquarters
(GCHQ)) in their informal report Soliloquy: A Cautionary Tale. The
introduction of lattice in cryptography brought in new, enthusiastic
researchers and propelled experienced researchers to the forefront
whose advice, often repeated was misunderstood and in many instances
ignored. It is worth explicitly stating that cryptography is
inextricably, and defined by its relationship with security objectives
and privacy. These constraints not only benefit from the
diversification schemes from their original construction, they demand
it. However, as lattice cryptography increased in popularity the demand
for diversification was so often ignored by researchers in favor of
achieving efficiency it became increasingly difficult to find
constructions which had not been constructed over a particular number
field; more over research had focused not simply to this number field
it had narrowed to the point of most constructions being defined
specifically over the cyclotomic field of characteristic two. It is
perhaps lucky that the discovery by the researchers at CESG was not
only disclosed publicly (the agency does not in general reveal much
less publish their findings), but that the defect was not found as a
consequence of an exploit, or even more at some point in the future
when cryptosystems having this type of construction may be widely
implemented and assumed secure. Introducing the Players Traditionally
cryptography has been a two player game between cryptographers and
cryptanalysts. However, cryptanalysis has diversified over time leaving
cryptographers to play defense against an increasingly diverse set of
cryptanalysts on the offense. Characterization of Cryptanalysts We
divide cryptanalysts into four subclasses given as a variation of the
subclasses of cryptographic secrecy. We make our division in reverse
order in terms of hardness as follows: Practical Secrecy: In this
subclass of cryptanalysts we have those which most frequently occur.
These cryptanalysts are in possesion of finite computational resources
but may use social, sidechannel, or other methods to obtain a break on
a system which does not break the mathematics of the scheme itself, may
not in the future be reproduced, but are sucessful in their attempt
none the less. Computational Secrecy: This subclass of cryptanalysts
are those who we can assume have at their disposal more resources and
are capable of more large scale attacks than those in the previous
subclass. They are still in possession of all the methods of the
practical cryptanalysts but we make the assumption that they are not
working alone but as a well organized unit such as those of a large
organization. These attacks are becoming more and more frequent the
organization itself may be known to the defense but for various reasons
( including but not limited to political) a direct confrontation must
be avoided. Unconditional Secrecy: In this subclass we find those
cryptanalysts who directly attack the mathematics of a given
cryptosystem. Specifically we have the cryptanalysts who play against
most modern systems based on the discrete logarithm problem, the
integer factorization problem, and other number theoretic schemes. We
limit this group to problems which are played over systems whose proofs
of security are given over schemes an average-case hardness. Perfect
Secrecy: This is the subclass of cryptanalysts we will be concerned
with for the remainder of this paper. These players are concerned with
breaking the mathematics of schemes based on intractible problems with
a provable security based on worst-case hardness assumptions.
Specifically, we concern ourselves with the concept of quantum-safe
cryptographic schemes over lattices. The Soliloquy Problem In 2010 at
the International Public Key Cryptography conference Smart and
Veracauteren introduced a cryptographic scheme [Sma2010] based on
principal ideal lattices with relatively small key sizes and promising
to enable FHE. In late 2014 GCHQ announced that they had developed an
equivalent cryptosystem to [Sma2010] in 2007 named Soliloquy, but from
2010 to 2013 they had constructed a successful attack against the
system and decided to cease development. Furthermore, they claimed
(without proof) that the assumed hard problem of finding the short
generator of a principal ideal lattice was both easy and efficient to
solve. This conflicted with the [Sma2010] claim that the problem of
finding a short generator was hard. In addition the solvability claim
GCHQ was not only given for quantum computers but also for ordinary
modern computer system with a practical amounts of resources. The
announcement by GCHQ created quite a stir in the cryptographic
community. The concern was due to the seeming quantum-safe properties
of lattice based cryptosystems. Since, ideal lattice-based schemes are
a small subset of these schemes there was a need to ensure that the
attack did not imply similar weaknesses in the more general schemes or
any other special cases of lattice-based schemes. Summary of Contents
First Half In the first half of this paper we give an overview of fully
homomorphic encryption (FHE) as well as describe the need for such
systems. We give a summary of the history of FHE schemes and an
overview of the first sucessful construction given by Gentrys Stanford
PhD thesis [Gen2009] as well as the construction from [Sma2010]. We
contrast these two schemes with an overview of the more general
construction of Peikerts RLWE and discuss the LWE problem, and the
concept of the mathematical basis of hardness assumptions. We then give
an overview of the quantum and general solutions proved since the GCHQ
announcement and summarize the results found by researchers in the
process of verification of these claims. Second Half In the second half
of this paper we discuss the problem of verifying if RLWE schemes come
with a connected expander graph and its implication that FHE is circuit
private. Background Cryptographic Construction Types Cryptographic
Primitives Does this go here or under PKC Public Key Cryptography
Problem solved by PKC Public Key Cryptography (PKC) solves the problem
of how to enable two people who have never met to communicate securely
over insecure channels. PKC uses a one-way function as its public key;
this function is easy to compute but hard to invert and is available
publicly for use as an encryption method for the key owner. A second
key called the private key is known only to the key owner. The private
key is a trapdoor function which can invert a ciphertext back to its
plaintext form. PKC consists of two directions the forward direction is
the one-way public key which easily computes the ciphertext, the
backward direction is the trapdoor private key, which easily inverts to
the plaintext. PKC Diagram (General) Asymmetric & Symmetric
Cryptography The concepts of symmetric and asymmetric cryptographic-key
schemes are given below as independent forms. However, it is rarely the
case that these schemes will be applied in this manner particularly in
a Public-key asymmetric system. Symmetric-key Cryptography In
cannonical symmetric-key schemes only one key is generated for both the
encryption and decryption algorithms. Symmetric-keys must therefore be
exchanged over secure channels where both communicating parties have
agreed in advance to the method of encryption and have exchanged secret
keys. Formally, we say the two parties are in possesion of a shared
secret and each party is mutually, as well as equally responsible for
the maintenance of the secret which establishes and secures their
communications. In certain contexts this method can be made to be as
secure as an asymmetric key system its main drawback is the shared
secret which has the ability to invalidate the integrity of
communications if either party is compromised. Additionally since the
method requires a secure channel as well as a shared secret both
parties must know they will have a need for encrypted communication in
advance and negotiate the means of transmission and key exchange prior
to communicating. This means the parties must know each other prior to
communicating. Asymmetric-key Cryptography In contrast symmetric-key
schemes are determined by two sets of keys one for each party. Each set
consists of a private key known only to the owner and a public key
which can be widely distributed. It seems at first glance that
asymmetric and symmetric cryptosystems are isolated methods of
encryption. However, symmetric encryption is well-suited for the task
of assisting asymmetric key schemes not only for efficiency purposes
but also in the role of key management. Provable Security Should
possibly merge this section and the Hardness Assumptions section
Defining Security with Games Semantic Security: Mental Poker and
Partial Information Semantic Security was a game proposed by Goldwasser
and Micali in their 1982 paper entitled: Probabilistic Encryption & How
to Play Mental Poker Keeping Secret All Partial Information" Citation:
@Gol1982 @inproceedings{Goldwasser:1982:PEA:800070.802212, author =
{Goldwasser, Shafi and Micali, Silvio}, title = {Probabilistic
Encryption \&Amp; How to Play Mental Poker Keeping Secret All Partial
Information}, booktitle = {Proceedings of the Fourteenth Annual ACM
Symposium on Theory of Computing}, series = {STOC '82}, year = {1982},
isbn = {0-89791-070-2}, location = {San Francisco, California, USA},
pages = {365--377}, numpages = {13}, url =
{http://doi.acm.org/10.1145/800070.802212}, doi =
{10.1145/800070.802212}, acmid = {802212}, publisher = {ACM}, address =
{New York, NY, USA}, } The Mental Poker game was based on the
implementation of the Diffie-Hellman implementation of RSA by Rivest,
Shamir, Adelman, and Rabin. (See footnote below) Footnote:
Specifically, @Gol1982 cite Rabins 1979 Technical Memo Digitalized
Signatures and Public-key Functions as Intractable as Factorization
@Rab1979 And, Rivest, Shamir, and Adelmans paper from Publications of
the ACM, February 1978 A Method for Obtaining Digital Signatures and
Public Key Cryptosystems. @Riv1978 Formal Definition: @Gol1982 Proposed
the following property for any implementation of a Diffie-Hellman
Public-key Cryptosystem: An adversary, who knows the encryption of an
algorithm and is given the cypher text, cannot obtain any information
about the cleartext." @Gol1982 Pg.1, Paragraph 1, Lines 3-6 Informally:
A given cryptographic scheme is considered insecure if it is possible
for an adversary to recover any information about the plaintext, using
the ciphertext, but without knowing the private key. That is, if it is
feasible for the adversary to find out some information about the
plaintext of the message or recover useful information about the
plaintext of the message by manipulating the ciphertext in a reasonable
amount of time, (before he dies of old age or the heat death of the
universe occurs) then the cryptosystem that created those ciphertext
messages (is broke as shit) is insecure. Kristi fix this wtf dude…
Weaknesses in the Assumptions of RSA Goldwasser and Micali pointed out
that the security assumptions given in @Riv1978 and @Rab1979 had some
particularly significant weaknesses that could not be assumed lightly.
Namely, Assumption 1: Security of the RSA system is based on the
intractability of the number theoretic problems of factorization, index
finding, and the decision problem of whether or not a number is a
quadratic residue with respect to a composite modulus. This assumption
states that the impossible hardness of one of these problems is
equivalent to RSA being computationally infeasible. Assumption 2: The
assumption that there exists a trapdoor function f(x) that is easily
computed, while x is not easily computed from f(x) unless some
additional information is known. Not all hard problems are as hard as
other hard problems: RSA is based on the assumption that factoring
large composites is a hard problem. However, not all large composite
numbers are hard to factor in fact some are quite easy. For example, if
c is a composite such that c=qp for prime p and q Equivalent and
Stronger Definitions for Security CPA IND-CPA says if an attacker can
choose any plaintext and obtain the corresponding ciphertext, then if
the system is secure this information does not help them find the
private key. However, the issue with CPA is that it depends on the
choices of an adversary who is unaware of the secret key. i.e. If the
attacker has two messages then they have no idea if either one contains
the key since they don't know what the key is. The problem that happens
is when I decided to encrypt my own key. Then by CPA I could encrypt
the key and the scheme return my key unencrypted and it still satisfies
CPA, and even though the scheme has blatantly revealed my private key,
technically it is still semantically secure under CPA (except I am not
attacking my own accounts and I had better know my own key). CCAI CCAII
InfoSec & it's Objectives Their are four basic security objectives that
must be considered when constructing any system concerned with securing
data or information. The four basic definitions allow for the
derivation of all other security objectives which may or may not be
necessary to ensure the security of information for a given system.
Confidentiality: The objective of confidentiality ensures that
unauthorized users will not be purposefully or accidentally give access
to the resources protected by a given security system. Integrity:
Ensures that the resources are preserved, used, and appropriately
maintained throughout their life-cycle under the system. That is, any
data is not alterable in an undetectable manner, retains the same
accuracy as its created date (or registered modification date), and is
complete with respect to its creation and activity log. Availability:
Non-repudiation: Security Reduction Definition Cryptographic Security
Audit Add check list to Appendix Hardness Assumptions Average-case We
say a cryptographic algorithm or cryptosystem has average-case hardness
if: Informally: The mathematical problem the algorithm (or system of
algorithms) is based on cannot be solved in polynomial time by a
standard computer with a slightly more than reasonable amount of
resources. More-formally: If a key is chosen at random, then there is
no probabilistic polynomial time algorithm that solves the mathematical
problem for some non negligible probability. Best-case Worst-case For
an average-case security proof we are assured the mathematical problem
the scheme is based on cannot be solved by a classical computer system
in polynomial time in all except a negligible set of its random
functional instances. A worst-case security proof carries all the
security given by the average-case, but it also covers the small set of
functional instances the average-case counts as negligible. Informally,
a worst-case security proof takes the functional instance which is
hardest to solve and shows the scheme is as least as strong as that
instance, and since there are no harder instances, there does not exist
a computer which can solve its mathematical problem in polynomial time.
Post-quantum cryptography The idea of quantum-safe Need for
quantum-safety Schemes with provably secure claims which may or have
been or could be efficient upon implementation Lattice-base
Cryptography Benefits in common with other post-quantum cryptographic
schemes Lattice-based Cryptosystems are a subclass of Post-quantum
cryptosystems. A worst-case security proof for a lattice based scheme
not only claims security against attacks from classical (current)
computer systems, but quantum as well. Drawbacks Not as efficient (yet)
as other more well known schemes. Unique benefit of LBC FHE Fully
Homomorphic Encryption What is FHE Homomorphic Properties of
Cryptographic Schemes Many classical public key cryptography methods
have a homomorphic property, this property says if add two things
together then their sum contains information about those two things.
That is they use successive operations of addition, multiplication,
etc. to create ciphertext. Informal Definition (verbal) Formal
Definition (symbolic) We call these schemes Partially Homomorphic or
Doubly Homomorphic since their operations can be performed
simultaneously. Problem Solved by FHE The problem which FHE solves is
only slightly different than that of PKC. We want to take our encrypted
data, send it to an unauthorized party for processing, and receive the
same results we would get if we had never encrypted it.

FHE schemes are not limited to a single operation and can
simultaneously compute any number of arbitrary operations. The
cyclotomic field of characteristic two infers so many convenient
structures it is often chosen without consideration to any other ring.
Recent by GCHQ in their own construction, and verified by Peikert
(GATech) in the later part of this year; is that for these rings the
mere guarantee of a short basis reduces the security of the
construction to the easy case, allowing the lattice to be efficiently
decoded, leaving the private key exposed. All cryptographic schemes
such as RSA, El Gamal, etc. must satisfy the requirements of semantic
security, specifically they must satisfy the Indistinguishable Chosen
Plaintext Attack (IND-CPA) requirement. Cryptologists have spent a
great deal of their time removing the natural homomorphisms from most
schemes like RSA. While at the same time wishing for a doubly
homomorphic system. So why would anyone want a doubly homomorphic
encryption system? A doubly homomorphic encryption system would mean I
can use binary to derive boolean logic circuits, if I can do this I can
write any arbitrary program to run on encrypted bits (even Mario Cart).
However, there are situations which can occur where the semantic
security definition is satisfied but an insecurity still occurs. One
such instance is the case where you encrypt your own private key. This
situation is called circular security, and it actually happens pretty
often.

Define 
Example 
Use Cases 
Cloud Outsourcing 
Ajtai's Discovery 
Learning with Errors (LWE) 
LWE, has a quantum reduction from SVP. LWE forms the basis of security for many other algorithms which gain SVP as basis for their own security. [SV10] is a special case. While ideal rings provide many benefits due to their structure, the choice of ring is as
important as the choice of base problem. 
Definition 
Example 
Regev's Average-Worst Case Hardness Connection 
Ring LWE 
The additional application areas and ease of implementation due to ideal ring structure is convenient, the choice of ring its self is as important as the problem it is based upon. An
advantage of Ring-LWE is its much smaller set of keys; the ring allows for more efficient and larger ranges in application areas. 
Ideal Lattice-based Cryptography 
Ideal Ring-LWE allows an even broader range of applications and efficiency than RLWE due to the additional structure provided by the principal ideals. 
Gentrys Scheme [Gen2009] 
Provides a contrast in terms of efficiency for models following it. This allows us a convenient way to show the reader how quickly the research progresses in this area. 
[Gen2009] Was the first successful construction fully homomorphic encryption scheme In 2009 Craig Gentry constructed the first cryptographic system capable of evaluating arbitrary operations. The scheme was constructed over a principal ideal lattice [Gen2009] uses a bootstrap method to prove its security.
Definition: Bootstrap Method 
Smarts Scheme 
Peikerts Scheme 
Greedy Problem Management 
Insert Quote by Regev about Principal Ideal Lattice schemes being greedy 
Discuss Efficiency/Security Balancing
FORMAT
