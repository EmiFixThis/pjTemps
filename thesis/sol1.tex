\section{Soliloquy Construction}

Let $n \in \mathbb{P}$ : $|n| \approx 10$ bits.
Let $\zeta$ be a primitive $n^{th}$ root of unity.

We define $K$ to be the number field $\mathbb{Q}(\zeta)$ and $\mathcal(O)$ the ring of integers $\mathbb[\zeta] \in K$. 

Select $\alpha$ as a candidate for $pk = \sum_{i=1}^{n} (\alpha_{i} \zeta_{i}) \in \mathcal{O}$, such that the coefficients $\alpha_{i}$ are taken from a discrete Gaussian distribution with a mean of zero.  
i.e. The coefficients are relatively small. 

Let $\mathcal{p}=\norm{\alpha}$.

The conditions under which an ordered pair, ($\alpha, \mathcal{p}$) can be used as a set of Soliloquy keys ($pk$, $sk$), are 
\begin{itemize}
\item $\mathcal{p} \in \mathbb{P}$  and 
\item $c=2^{\frac{p-1}{n}} \not\equiv 1(mod \mathcal{p})$.
\item Then the quantity $c$ occurs with probability $1-\frac{1}{n}$. 
\end{itemize}

Using these conditions we guarantee that we have at least one principal ideal of the form \\ $\mathcal{P} = \mathcal{pO}+(\zeta-c)\mathcal{O}$, where a prime $\mathcal{p} \equiv 1(mod n)$ is the principal ideal $\mathcal{pO}$ \\ for a particularly chosen $c=2^{\frac{(p-1)}{n}}$. 

If $\mathbb{pO}$ is a product of prime ideals $\mathcal{P}_{i}$ with representation $\mathcal{P}_{i} = \mathcal{pO} + (\zeta-c_{i}) \mathcal{O}$ where the $c_{i}$ are the non-trivial $n_{th}$ of unity mod $p$. 

Since the Galois group $Gal(K/\mathbb{Q})$ gives a permutation of prime ideals $\mathcal{P}_{i}$, then by a reordering of the coefficients $a_{i}$, (i.e. taking some Galois conjugate of $\alpha$), we can ensure $\alpha \mathcal{O} = \mathcal{P}$. 

