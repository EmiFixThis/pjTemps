\section{Computational Complexity} 

\todo{this section to be merged or deprecated}
\subsection{Average-case}  
We say a cryptographic algorithm or cryptosystem has average-case hardness if: 
Informally: The mathematical problem the algorithm (or system of
algorithms) is based on cannot be solved in polynomial time by a
standard computer with a slightly more than reasonable amount of resources. 
\newline
More-formally: If a key is chosen at random, then there is no probabilistic polynomial time algorithm that solves the mathematical problem for some non-negligible probability. 


\subsection{Worst-case} 
For an average-case security proof we are assured the mathematical problem
the scheme is based on cannot be solved by a classical computer system in polynomial time in all except a negligible set of its random functional instances.
A worst-case security proof carries all the security given by the average-case, but it also covers the small set of functional instances the average-case counts as negligible. 
\newline
Informally, a worst-case security proof takes the functional instance which is
hardest to solve and shows the scheme is as least as strong as that
instance, since there are no harder instances, there does not exist
a computer which can solve its mathematical problem in polynomial time.