\documentclass[10pt]{article}
\usepackage{amsmath}
\usepackage{tcolorbox}
\usepackage{amssymb}

\makeatletter
\def\imod#1{\allowbreak\mkern10mu({\operator@font mod}\,\,#1)}
\makeatother

\begin{document}
%\subsection{Soliloquy Construction}
\\
\begin{tcolorbox}
\subsection*{Encryption \& Decryption}
\\ \\
\subsubsection*{Encryption}
\\
The encryption function is:
$$ \psi(\epsilon) = \psi \left( \sum_{i=1}^{n}e_{i}c^{i} \right) = \sum_{i=0}^{n} e_{i}c^{i}\imod{p}=z. $$
\\
The ephemeral\footnote{Emphemeral keys are key which are (usually)  generated each time the key generation process is executed. } key $\epsilon$ is sampled from the ring of integers $\mathcal{O}$, where the coefficients are sampled with a mean value of zero, the encrypted output $z$, is a positive rational number.
\\ \\
\subsubsection*{Decryption}
\\
To decrypt $\epsilon$, we use the closest integer function $\epsilon = \lceil z \rfloor \lceil z\alpha^{-1}\cdot \alpha \rfloor$, which gives $\epsilon$ small enough that $\lceil \epsilon \cdot \alpha^{-1}\rfloor = 0$. \\
\end{tcolorbox}
\end{document}
